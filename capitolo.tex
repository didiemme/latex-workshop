\chapter{Introduzione} 
\label{Cap1}

\section{Background}
Scopo della Pattern recognition È di classificare i dati (pattern) basandosi su una conoscenza a priori o su informazioni statistiche estratte 
dai pattern stessi. I dati da classificare sono solitamente insiemi di misurazioni o osservazioni, che rappresentano i punti in uno spazio 
multidimensionale.
\subsection{Outline}
L'argomento verrà presentato seguendo quest'ordine:
\begin{itemize}
\item attraverso il capitolo \ref{Cap2} verrà descritto il problema, verranno trattate alcune soluzioni storiche e descritte in 
modo particolareggiato alcune soluzioni che descrivono l'odierno stato dell'arte.
\item il capitolo \ref{Cap3} È dedicato all'analisi del dominio caratterizzato dalle ceramiche Kamares e verrà descritto il software sviluppato
e i problemi incontrati nella realizzazione.
\item etc...
\end{itemize}